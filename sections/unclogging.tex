Sometimes restricting privileges on lab machines is not enough to prevent students from clogging the machines.  If not locked down sufficiently, students can potentially cause headache to system administrators.  Often, the students are not even malicious.  They could simply be doing assignments that have the potential to overload the lab machines.

For example, in an operating systems course, students must learn how to properly create, fork, kill, and end processes.  As with any programming, practice is needed to properly learn a new programming technique and many mistakes will be made along the way with each program run.  With operating systems concepts, many leftovers will remain after each program run.  Forking of processes will leave behind orphaned children if the program does not exit cleanly.  In addition, if a student's loop runs away, the potential for overloading the system with forked processes exists.  

To prevent this, steps need to be taken to limit what a student can do.  For example, a student can run through a finite set of process PIDs very quickly with a loop.  If the system does not have a limit for a single user to have, the user can make the system run out of PIDs.  By default, some Linux distributions set the kernel limit for a single user to 1024.  However, on other distributions and other types of Unix, this limit is not necessarily in place or is too high.  Thus, it is important to check this and set it accordingly.  It may be a good idea to drop the process count to below a hundred if the student does not require a window manager (which could also be useful for forcing students to learn good command line habits).  

In addition to PIDs, students can still clog the machine by using too much CPU, memory, or disk space.  CPU and memory can both be limited by editing kernel parameters by user (by default, they are unlimited).  Disk space can be limited by setting a quota on home directories.  However, users may still be able to write to temp directories and fill those up, so it is also useful to mount those directories from separate partitions of limited size.  Temp directories may still be important for operating system performance, so a cronjob to delete excess temp files may be necessary.  In the case of CECS labs, each machine can be reimaged within 10-15 minutes, so it may be simpler from an administration perspective to just let the students use the full resources of lab machines (minus administrative privileges) to use in programming.  

For many of these issues, the answer is simply to schedule a reboot every night.  This will clear all orphaned processes, interrupt any processes that could be using resources, and reset the environment.  This is very simple to do by make an entry in root's crontab.  The reboots should take place in the middle of the night during less used hours to prevent loss of work.  

This may create a few issues on some Linux distributions though.  On Ubuntu, the file system can get somewhat dirty after a few weeks and on reboot will mount the file system read-only if it detects errors.  To prevent this, we set it to do an fsck (file system check) if errors exist.  Delayed login mode is required for this though, but since students are prevented from manually rebooting the machines, this will not affect much.  It also helps that the reboots take place at night.  Lastly, the setup should not require and administrator to interact if problems are found.  We have set fsck to fix errors automatically without asking for confirmation.  

Verbose mode is also very helpful for troubleshooting machines quickly (especially while classes are in session).  A splash screen is nice when trying to hide confusing output from non-technical users, but in a room full of computer science students, this really is not necessary.  

